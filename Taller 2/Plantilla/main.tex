\documentclass[12pt,a4paper]{article}
\usepackage[spanish]{babel}
\usepackage[utf8]{inputenc}
\usepackage{amsmath}
\usepackage{array}
\usepackage{geometry}
\geometry{margin=2.5cm}

\title{\textbf{Segundo Trabajo Estadística II} \\[0.5cm]
Pruebas de Hipótesis}
\author{
Integrantes: \\ 
1. Juan David Mena Gamboa \\ 
2. Luis Angel Bolaño Lopez \\ 
3. Heyner}
\date{\today}

\begin{document}
\maketitle

\section*{Ejercicio 1: Prueba de hipótesis para la media de una muestra con varianza conocida}
\begin{itemize}
    \item \textit{Planteamiento del problema:} Se desea evaluar si el ahorro promedio por criar animales en la población difiere significativamente de 100,000 pesos.
    \item Hipótesis: \quad $H_0: \mu = 100,000$ \hspace{2cm} $H_1: \mu \neq 100,000$
    \item Nivel de significancia: $\alpha = 0.05$
    \item Estadístico de Prueba: Z
\end{itemize}

\begin{tabular}{|m{7cm}|m{7cm}|}
\hline
\textbf{Estadístico Empleado} & Z para media con varianza conocida \\ \hline
\textbf{Fórmula del estadístico} & $Z = \frac{\bar{X} - \mu_0}{\sigma/\sqrt{n}}$ \\ \hline
\textbf{Nombre Parámetro} & \textbf{Valor} \\ \hline
Media poblacional bajo $H_0$ ($\mu_0$) & 100,000 \\ \hline
Desviación estándar poblacional ($\sigma$) & 500,000 \\ \hline
Tamaño de muestra (n) & 4,080 \\ \hline
Media muestral ($\bar{X}$) & 159,471.37 \\ \hline
Valor Estadístico & 7.597 \\ \hline
Valor P & $< 0.001$ \\ \hline
\end{tabular}

\begin{itemize}
    \item Regla de decisión: Rechazar $H_0$ si $|Z| > 1.96$ o si p-valor $< 0.05$
    \item Decisión: Rechazar $H_0$
    \item Conclusión: Existe evidencia estadística suficiente para concluir que el ahorro promedio por criar animales en la población difiere significativamente de 100,000 pesos. El ahorro promedio observado (159,471.37 pesos) es significativamente mayor al valor hipotético.
\end{itemize}

\section*{Ejercicio 2: Prueba de hipótesis para la proporción de una muestra}
\begin{itemize}
    \item \textit{Planteamiento del problema:} Se desea evaluar si la proporción de personas que asistieron a reuniones familiares difiere del 50\%.
    \item Hipótesis: \quad $H_{0}: p = 0.5$ \hspace{2cm} $H_{1}: p \neq 0.5$
    \item Nivel de significancia: $\alpha = 0.05$
    \item Estadístico de Prueba: Z
\end{itemize}

\begin{tabular}{|m{7cm}|m{7cm}|}
\hline
\textbf{Estadístico Empleado} & Z para proporción \\ \hline
\textbf{Fórmula del estadístico} & $Z = \frac{\hat{p} - p_0}{\sqrt{p_0(1-p_0)/n}}$ \\ \hline
\textbf{Nombre Parámetro} & \textbf{Valor} \\ \hline
Proporción bajo $H_0$ ($p_0$) & 0.5 \\ \hline
Tamaño de muestra (n) & 166,341 \\ \hline
Proporción muestral ($\hat{p}$) & 0.0175 \\ \hline
Error estándar (SE) & 0.0012 \\ \hline
Valor Estadístico & -393.54 \\ \hline
Valor P & $< 0.001$ \\ \hline
\end{tabular}

\begin{itemize}
    \item Regla de decisión: Rechazar $H_0$ si $|Z| > 1.96$ o si p-valor $< 0.05$
    \item Decisión: Rechazar $H_0$
    \item Conclusión: Existe evidencia estadística suficiente para concluir que la proporción de personas que asistieron a reuniones familiares difiere significativamente del 50\%. La proporción observada (1.75\%) es significativamente menor al valor hipotético.
\end{itemize}

\section*{Ejercicio 3: Prueba de hipótesis para la media de una muestra con varianza desconocida}
\begin{itemize}
    \item \textit{Planteamiento del problema:} Se desea evaluar si el valor promedio mensual por prácticas o pasantías difiere de 200,000 pesos.
    \item Hipótesis: \quad $H_{0}: \mu = 200,000$ \hspace{2cm} $H_{1}: \mu \neq 200,000$
    \item Nivel de significancia: $\alpha = 0.05$
    \item Estadístico de Prueba: t
\end{itemize}

\begin{tabular}{|m{7cm}|m{7cm}|}
\hline
\textbf{Estadístico Empleado} & t para media con varianza desconocida \\ \hline
\textbf{Fórmula del estadístico} & $t = \frac{\bar{X} - \mu_0}{s/\sqrt{n}}$ \\ \hline
\textbf{Nombre Parámetro} & \textbf{Valor} \\ \hline
Media poblacional bajo $H_0$ ($\mu_0$) & 200,000 \\ \hline
Tamaño de muestra (n) & 336 \\ \hline
Grados de libertad (df) & 335 \\ \hline
Media muestral ($\bar{X}$) & 998,708.33 \\ \hline
Desviación estándar muestral (s) & 621,977.40 \\ \hline
Valor Estadístico & 23.539 \\ \hline
Valor P & $< 0.001$ \\ \hline
\end{tabular}

\begin{itemize}
    \item Regla de decisión: Rechazar $H_0$ si $|t| > 1.967$ o si p-valor $< 0.05$
    \item Decisión: Rechazar $H_0$
    \item Conclusión: Existe evidencia estadística suficiente para concluir que el valor promedio mensual por prácticas difiere significativamente de 200,000 pesos. El valor promedio observado (998,708.33 pesos) es significativamente mayor al valor hipotético.
\end{itemize}

\section*{Ejercicio 4: Prueba de hipótesis para la diferencia de medias de dos muestras con varianzas conocidas}
\begin{itemize}
    \item \textit{Planteamiento del problema:} Se desea evaluar si existe diferencia entre el ahorro promedio por cultivar y por criar animales.
    \item Hipótesis: \quad $H_{0}: \mu_1 - \mu_2 = 0$ \hspace{2cm} $H_{1}: \mu_1 - \mu_2 \neq 0$
    \item Nivel de significancia: $\alpha = 0.05$
    \item Estadístico de Prueba: Z
\end{itemize}

\begin{tabular}{|m{7cm}|m{7cm}|}
\hline
\textbf{Estadístico Empleado} & Z para diferencia de medias con varianzas conocidas \\ \hline
\textbf{Fórmula del estadístico} & $Z = \frac{(\bar{X}_1 - \bar{X}_2) - 0}{\sqrt{\sigma_1^2/n_1 + \sigma_2^2/n_2}}$ \\ \hline
\textbf{Nombre Parámetro} & \textbf{Valor} \\ \hline
Media muestra 1 (cultivar) & 126,937.56 \\ \hline
Media muestra 2 (criar) & 159,471.37 \\ \hline
Desviación estándar 1 ($\sigma_1$) & 300,000 \\ \hline
Desviación estándar 2 ($\sigma_2$) & 500,000 \\ \hline
Tamaño muestra 1 ($n_1$) & 2,892 \\ \hline
Tamaño muestra 2 ($n_2$) & 4,080 \\ \hline
Diferencia de medias & -32,533.81 \\ \hline
Valor Estadístico & -3.385 \\ \hline
Valor P & 0.0007 \\ \hline
\end{tabular}

\begin{itemize}
    \item Regla de decisión: Rechazar $H_0$ si $|Z| > 1.96$ o si p-valor $< 0.05$
    \item Decisión: Rechazar $H_0$
    \item Conclusión: Existe evidencia estadística suficiente para concluir que hay diferencia significativa entre el ahorro promedio por cultivar y por criar animales. El ahorro por criar animales es significativamente mayor que por cultivar.
\end{itemize}

\section*{Ejercicio 5: Prueba de hipótesis para la diferencia de medias de dos muestras con varianzas iguales y desconocidas}
\begin{itemize}
    \item \textit{Planteamiento del problema:} Se desea evaluar si existe diferencia entre el valor promedio de prácticas y el ahorro promedio por cultivar, asumiendo varianzas iguales.
    \item Hipótesis: \quad $H_{0}: \mu_1 - \mu_2 = 0$ \hspace{2cm} $H_{1}: \mu_1 - \mu_2 \neq 0$
    \item Nivel de significancia: $\alpha = 0.05$
    \item Estadístico de Prueba: t (pooled)
\end{itemize}

\begin{tabular}{|m{7cm}|m{7cm}|}
\hline
\textbf{Estadístico Empleado} & t para diferencia de medias con varianzas iguales \\ \hline
\textbf{Fórmula del estadístico} & $t = \frac{(\bar{X}_1 - \bar{X}_2) - 0}{s_p\sqrt{1/n_1 + 1/n_2}}$ \\ \hline
\textbf{Nombre Parámetro} & \textbf{Valor} \\ \hline
Media muestra 1 (prácticas) & 998,708.33 \\ \hline
Media muestra 2 (cultivar) & 126,937.56 \\ \hline
Tamaño muestra 1 ($n_1$) & 336 \\ \hline
Tamaño muestra 2 ($n_2$) & 2,892 \\ \hline
Grados de libertad (df) & 3,226 \\ \hline
Varianza combinada ($s_p^2$) & 190,877,666,859 \\ \hline
Diferencia de medias & 871,770.77 \\ \hline
Valor Estadístico & 34.620 \\ \hline
Valor P & $< 0.001$ \\ \hline
\end{tabular}

\begin{itemize}
    \item Regla de decisión: Rechazar $H_0$ si $|t| > 1.961$ o si p-valor $< 0.05$
    \item Decisión: Rechazar $H_0$
    \item Conclusión: Existe evidencia estadística suficiente para concluir que hay diferencia significativa entre el valor promedio de prácticas y el ahorro promedio por cultivar. El valor de prácticas es significativamente mayor que el ahorro por cultivar.
\end{itemize}

\section*{Ejercicio 6: Prueba de hipótesis para la diferencia de medias de dos muestras con varianzas desiguales y desconocidas}
\begin{itemize}
    \item \textit{Planteamiento del problema:} Se desea evaluar si existe diferencia entre el valor promedio de prácticas y el ahorro promedio por criar animales, sin asumir varianzas iguales.
    \item Hipótesis: \quad $H_{0}: \mu_1 - \mu_2 = 0$ \hspace{2cm} $H_{1}: \mu_1 - \mu_2 \neq 0$
    \item Nivel de significancia: $\alpha = 0.05$
    \item Estadístico de Prueba: t de Welch
\end{itemize}

\begin{tabular}{|m{7cm}|m{7cm}|}
\hline
\textbf{Estadístico Empleado} & t de Welch para diferencia de medias \\ \hline
\textbf{Fórmula del estadístico} & $t = \frac{(\bar{X}_1 - \bar{X}_2) - 0}{\sqrt{s_1^2/n_1 + s_2^2/n_2}}$ \\ \hline
\textbf{Nombre Parámetro} & \textbf{Valor} \\ \hline
Media muestra 1 (prácticas) & 998,708.33 \\ \hline
Media muestra 2 (criar) & 159,471.37 \\ \hline
Tamaño muestra 1 ($n_1$) & 336 \\ \hline
Tamaño muestra 2 ($n_2$) & 4,080 \\ \hline
Grados de libertad (df) & 373.42 \\ \hline
Diferencia de medias & 839,236.96 \\ \hline
Error estándar & 34,867.44 \\ \hline
Valor Estadístico & 24.069 \\ \hline
Valor P & $< 0.001$ \\ \hline
\end{tabular}

\begin{itemize}
    \item Regla de decisión: Rechazar $H_0$ si $|t| > 1.966$ o si p-valor $< 0.05$
    \item Decisión: Rechazar $H_0$
    \item Conclusión: Existe evidencia estadística suficiente para concluir que hay diferencia significativa entre el valor promedio de prácticas y el ahorro promedio por criar animales. El valor de prácticas es significativamente mayor que el ahorro por criar animales.
\end{itemize}

\section*{Ejercicio 7: Prueba de hipótesis para la diferencia de proporciones de dos muestras}
\begin{itemize}
    \item \textit{Planteamiento del problema:} Se desea evaluar si existe diferencia en la proporción de asistencia a reuniones familiares entre grupos de bajo y alto valor de prácticas.
    \item Hipótesis: \quad $H_{0}: p_1 - p_2 = 0$ \hspace{2cm} $H_{1}: p_1 - p_2 \neq 0$
    \item Nivel de significancia: $\alpha = 0.05$
    \item Estadístico de Prueba: Z
\end{itemize}

\begin{tabular}{|m{7cm}|m{7cm}|}
\hline
\textbf{Estadístico Empleado} & Z para diferencia de proporciones \\ \hline
\textbf{Fórmula del estadístico} & $Z = \frac{(\hat{p}_1 - \hat{p}_2) - 0}{\sqrt{\hat{p}(1-\hat{p})(1/n_1 + 1/n_2)}}$ \\ \hline
\textbf{Nombre Parámetro} & \textbf{Valor} \\ \hline
Proporción grupo bajo ($\hat{p}_1$) & 0.0138 \\ \hline
Proporción grupo alto ($\hat{p}_2$) & 0.0085 \\ \hline
Tamaño grupo bajo ($n_1$) & 145 \\ \hline
Tamaño grupo alto ($n_2$) & 118 \\ \hline
Proporción combinada ($\hat{p}$) & 0.0119 \\ \hline
Diferencia de proporciones & 0.0053 \\ \hline
Valor Estadístico & 0.427 \\ \hline
Valor P & 0.6697 \\ \hline
\end{tabular}

\begin{itemize}
    \item Regla de decisión: Rechazar $H_0$ si $|Z| > 1.96$ o si p-valor $< 0.05$
    \item Decisión: No rechazar $H_0$
    \item Conclusión: No existe evidencia estadística suficiente para concluir que hay diferencia significativa en la proporción de asistencia a reuniones familiares entre grupos de bajo y alto valor de prácticas.
\end{itemize}

\section*{Ejercicio 8: Prueba de hipótesis para muestras dependientes}
\begin{itemize}
    \item \textit{Planteamiento del problema:} Se desea evaluar si existe diferencia sistemática entre el ahorro por cultivar y por criar animales en las mismas personas.
    \item Hipótesis: \quad $H_{0}: \mu_d = 0$ \hspace{2cm} $H_{1}: \mu_d \neq 0$
    \item Nivel de significancia: $\alpha = 0.05$
    \item Estadístico de Prueba: t pareado
\end{itemize}

\begin{tabular}{|m{7cm}|m{7cm}|}
\hline
\textbf{Estadístico Empleado} & t para muestras pareadas \\ \hline
\textbf{Fórmula del estadístico} & $t = \frac{\bar{d} - 0}{s_d/\sqrt{n}}$ \\ \hline
\textbf{Nombre Parámetro} & \textbf{Valor} \\ \hline
Número de pares (n) & 1,296 \\ \hline
Grados de libertad (df) & 1,295 \\ \hline
Media de diferencias ($\bar{d}$) & 9,776.68 \\ \hline
Desviación estándar de diferencias ($s_d$) & 575,782.40 \\ \hline
Error estándar de diferencias & 15,993.96 \\ \hline
Valor Estadístico & 0.611 \\ \hline
Valor P & 0.5411 \\ \hline
\end{tabular}

\begin{itemize}
    \item Regla de decisión: Rechazar $H_0$ si $|t| > 1.962$ o si p-valor $< 0.05$
    \item Decisión: No rechazar $H_0$
    \item Conclusión: No existe evidencia estadística suficiente para concluir que hay diferencia sistemática entre el ahorro por cultivar y por criar animales en las mismas personas.
\end{itemize}

\section*{Ejercicio 9: Prueba de bondad de ajuste}
\begin{itemize}
    \item \textit{Planteamiento del problema:} Se desea evaluar si la distribución observada de asistencia a reuniones familiares se ajusta a una distribución uniforme (50\%-50\%).
    \item Hipótesis: \quad $H_{0}:$ La distribución observada se ajusta a la esperada \hspace{1cm} $H_{1}:$ La distribución observada no se ajusta a la esperada
    \item Nivel de significancia: $\alpha = 0.05$
    \item Estadístico de Prueba: $\chi^2$
\end{itemize}

\begin{tabular}{|m{7cm}|m{7cm}|}
\hline
\textbf{Estadístico Empleado} & Chi-cuadrado de bondad de ajuste \\ \hline
\textbf{Fórmula del estadístico} & $\chi^2 = \sum \frac{(O_i - E_i)^2}{E_i}$ \\ \hline
\textbf{Nombre Parámetro} & \textbf{Valor} \\ \hline
Número de categorías (k) & 2 \\ \hline
Grados de libertad (df) & 1 \\ \hline
Frecuencia observada "No" ($O_1$) & 163,423 \\ \hline
Frecuencia observada "Sí" ($O_2$) & 2,918 \\ \hline
Frecuencia esperada "No" ($E_1$) & 83,170.5 \\ \hline
Frecuencia esperada "Sí" ($E_2$) & 83,170.5 \\ \hline
Valor Estadístico & 154,873.75 \\ \hline
Valor P & $< 0.001$ \\ \hline
\end{tabular}

\begin{itemize}
    \item Regla de decisión: Rechazar $H_0$ si $\chi^2 > 3.841$ o si p-valor $< 0.05$
    \item Decisión: Rechazar $H_0$
    \item Conclusión: Existe evidencia estadística suficiente para concluir que la distribución observada de asistencia a reuniones familiares no se ajusta a una distribución uniforme. La proporción real de asistencia (1.75\%) difiere significativamente del 50\% esperado.
\end{itemize}

\section*{Ejercicio 10: Prueba de independencia}
\begin{itemize}
    \item \textit{Planteamiento del problema:} Se desea evaluar si existe asociación entre la asistencia a reuniones familiares y el nivel de valor de prácticas (agrupado en terciles).
    \item Hipótesis: \quad $H_{0}:$ Las variables son independientes \hspace{2cm} $H_{1}:$ Las variables están asociadas
    \item Nivel de significancia: $\alpha = 0.05$
    \item Estadístico de Prueba: $\chi^2$
\end{itemize}

\begin{tabular}{|m{7cm}|m{7cm}|}
\hline
\textbf{Estadístico Empleado} & Chi-cuadrado de independencia \\ \hline
\textbf{Fórmula del estadístico} & $\chi^2 = \sum \frac{(O_{ij} - E_{ij})^2}{E_{ij}}$ \\ \hline
\textbf{Nombre Parámetro} & \textbf{Valor} \\ \hline
Número de filas (r) & 3 \\ \hline
Número de columnas (c) & 2 \\ \hline
Grados de libertad (df) & 2 \\ \hline
Frecuencias observadas & Bajo: 125,0; 2,1 \\ \hline
& Medio: 101,0; 1,1 \\ \hline
& Alto: 106,0; 1,1 \\ \hline
Frecuencias esperadas & Bajo: 125.5, 1.5 \\ \hline
& Medio: 100.8, 1.2 \\ \hline
& Alto: 105.7, 1.3 \\ \hline
Valor Estadístico & 0.257 \\ \hline
Valor P & 0.8793 \\ \hline
\end{tabular}

\begin{itemize}
    \item Regla de decisión: Rechazar $H_0$ si $\chi^2 > 5.991$ o si p-valor $< 0.05$
    \item Decisión: No rechazar $H_0$
    \item Conclusión: No existe evidencia estadística suficiente para concluir que hay asociación entre la asistencia a reuniones familiares y el nivel de valor de prácticas. Las variables parecen ser independientes.
\end{itemize}

\section*{Ejercicio 11: Prueba de signos}
\begin{itemize}
    \item \textit{Planteamiento del problema:} Se desea evaluar si existe tendencia sistemática en las diferencias entre rankings de ahorro por cultivar y por criar animales.
    \item Hipótesis: \quad $H_{0}:$ No hay diferencia sistemática de signos \hspace{1cm} $H_{1}:$ Existe diferencia sistemática de signos
    \item Nivel de significancia: $\alpha = 0.05$
    \item Estadístico de Prueba: Prueba de signos
\end{itemize}

\begin{tabular}{|m{7cm}|m{7cm}|}
\hline
\textbf{Estadístico Empleado} & Prueba de signos para muestras pareadas \\ \hline
\textbf{Fórmula del estadístico} & Estadístico = min(pos, neg) \\ \hline
\textbf{Nombre Parámetro} & \textbf{Valor} \\ \hline
Número total de pares (n) & 1,295 \\ \hline
Diferencias positivas & 679 \\ \hline
Diferencias negativas & 616 \\ \hline
Empates & 1 \\ \hline
Estadístico (min) & 616 \\ \hline
Valor P & 0.0849 \\ \hline
\end{tabular}

\begin{itemize}
    \item Regla de decisión: Rechazar $H_0$ si p-valor $< 0.05$
    \item Decisión: No rechazar $H_0$
    \item Conclusión: No existe evidencia estadística suficiente para concluir que hay tendencia sistemática en las diferencias entre rankings de ahorro por cultivar y por criar animales. Las diferencias parecen ser aleatorias.
\end{itemize}

\end{document}