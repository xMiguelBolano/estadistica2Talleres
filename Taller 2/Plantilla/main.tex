\documentclass[12pt,a4paper]{article}
\usepackage[spanish]{babel}
\usepackage[utf8]{inputenc}
\usepackage{amsmath}
\usepackage{array}
\usepackage{geometry}
\geometry{margin=2.5cm}

\title{\textbf{Segundo Trabajo Estadística II} \\[0.5cm]
Pruebas de Hipótesis}
\author{
Integrantes: \\ 
1. [Nombre] \\ 
2. [Nombre] \\ 
3. [Nombre]}
\date{}

\begin{document}
\maketitle

\section*{Ejercicio 1: Prueba de hipótesis para la media de una muestra con varianza conocida}
\begin{itemize}
    \item \textit{Planteamiento del problema (hipótesis de trabajo):}
    \item Hipótesis: \quad $H_{0}:$ \hspace{2cm} $H_{1}:$
    \item Nivel de significancia:
    \item Estadístico de Prueba:
\end{itemize}

\begin{tabular}{|m{7cm}|m{7cm}|}
\hline
\textbf{Estadístico Empleado} &  \\ \hline
\textbf{Fórmula del estadístico} &  \\ \hline
\textbf{Nombre Parámetro} & \textbf{Valor} \\ \hline
 &  \\ \hline
 &  \\ \hline
 &  \\ \hline
 &  \\ \hline
Valor Estadístico &  \\ \hline
Valor P &  \\ \hline
\end{tabular}

\begin{itemize}
    \item Regla de decisión:
    \item Decisión:
    \item Conclusión:
\end{itemize}

% ----------- Copiar este bloque y cambiar el título para los demás ejercicios ------------ %

\section*{Ejercicio 2: Prueba de hipótesis para la proporción de una muestra}
% Aquí copias el mismo bloque de itemize + tabla

\section*{Ejercicio 3: Prueba de hipótesis para la media de una muestra con varianza desconocida}

\section*{Ejercicio 4: Prueba de hipótesis para la diferencia de medias de dos muestras con varianzas conocidas}

\section*{Ejercicio 5: Prueba de hipótesis para la diferencia de medias de dos muestras con varianzas iguales y desconocidas}

\section*{Ejercicio 6: Prueba de hipótesis para la diferencia de medias de dos muestras con varianzas desiguales y desconocidas}

\section*{Ejercicio 7: Prueba de hipótesis para la diferencia de proporciones de dos muestras}

\section*{Ejercicio 8: Prueba de hipótesis para muestras dependientes}

\section*{Ejercicio 9: Prueba de bondad de ajuste}

\section*{Ejercicio 10: Prueba de independencia}

\section*{Ejercicio 11: Prueba de signos}

\end{document}
