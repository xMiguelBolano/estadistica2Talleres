\documentclass[12pt,a4paper]{article}
\usepackage[spanish]{babel}
\usepackage[utf8]{inputenc}
\usepackage{amsmath}
\usepackage{array}
\usepackage{geometry}
\geometry{margin=2.5cm}

\title{\textbf{Segundo Trabajo Estadística II} \\[0.5cm]
Pruebas de Hipótesis}
\author{
Integrantes: \\ 
1. Juan David Mena Gamboa \\ 
2. Luis Angel Bolaño Lopez \\ 
3. Heyner}
\date{\today}

\begin{document}
\maketitle

\section*{Ejercicio 1: Prueba de hipótesis para la media de una muestra con varianza conocida}
\begin{itemize}
    \item \textit{Planteamiento del problema:} La Gran Encuesta Integrada de Hogares (GEIH) 2024 del DANE recopila información sobre las actividades económicas de los hogares colombianos, incluyendo el ahorro generado por actividades agropecuarias. La variable de ahorro por criar animales (P3095S3) mide los ingresos netos obtenidos por esta actividad. Considerando que esta actividad representa una fuente importante de ingresos para muchas familias rurales y periurbanas, se cree que el ahorro promedio por criar animales en la población es igual a 100,000 pesos.
    \item Hipótesis: \quad $H_0: \mu = 100,000$ \hspace{2cm} $H_1: \mu \neq 100,000$
    \item Nivel de significancia: $\alpha = 0.030$
    \item Estadístico de Prueba: Z
\end{itemize}

\begin{tabular}{|m{7cm}|m{7cm}|}
\hline
\textbf{Estadístico Empleado} & Z para media con varianza conocida \\ \hline
\textbf{Fórmula del estadístico} & $Z = \frac{\bar{X} - \mu_0}{\sigma/\sqrt{n}}$ \\ \hline
\textbf{Nombre Parámetro} & \textbf{Valor} \\ \hline
Media poblacional bajo $H_0$ ($\mu_0$) & 100,000 \\ \hline
Desviación estándar poblacional ($\sigma$) & 500,000 \\ \hline
Tamaño de muestra (n) & 4,080 \\ \hline
Media muestral ($\bar{X}$) & 159,471.37 \\ \hline
Valor Estadístico & 7.598 \\ \hline
Valor P & $3.02 \times 10^{-14}$ \\ \hline
Valor Crítico & $\pm 2.170$ \\ \hline
\end{tabular}

\begin{itemize}
    \item Regla de decisión: Rechazar $H_0$ si $|Z| > 2.170$ o si p-valor $< 0.030$
    \item Decisión: Rechazar $H_0$
    \item Conclusión: Con un nivel de significancia de 0.030, la evidencia sugiere que el ahorro promedio por criar animales en la población no es igual a 100,000 pesos.
\end{itemize}

\section*{Ejercicio 2: Prueba de hipótesis para la proporción de una muestra}
\begin{itemize}
    \item \textit{Planteamiento del problema:} La GEIH 2024 incluye información sobre la participación social y familiar de los individuos. La variable de asistencia a reuniones familiares (P3087S1) captura la participación en eventos familiares, lo cual es un indicador importante del capital social y la cohesión familiar en Colombia. Dado que las reuniones familiares son tradicionalmente valoradas en la cultura colombiana, se cree que la proporción de personas que asistieron a reuniones familiares es igual al 50\%.
    \item Hipótesis: \quad $H_{0}: p = 0.5$ \hspace{2cm} $H_{1}: p \neq 0.5$
    \item Nivel de significancia: $\alpha = 0.070$
    \item Estadístico de Prueba: Z
\end{itemize}

\begin{tabular}{|m{7cm}|m{7cm}|}
\hline
\textbf{Estadístico Empleado} & Z para proporción \\ \hline
\textbf{Fórmula del estadístico} & $Z = \frac{\hat{p} - p_0}{\sqrt{p_0(1-p_0)/n}}$ \\ \hline
\textbf{Nombre Parámetro} & \textbf{Valor} \\ \hline
Proporción bajo $H_0$ ($p_0$) & 0.5 \\ \hline
Tamaño de muestra (n) & 166,341 \\ \hline
Proporción muestral ($\hat{p}$) & 0.0175 \\ \hline
Error estándar (SE) & 0.0012 \\ \hline
Valor Estadístico & -393.54 \\ \hline
Valor P & $< 0.001$ \\ \hline
Valor Crítico & $\pm 1.812$ \\ \hline
\end{tabular}

\begin{itemize}
    \item Regla de decisión: Rechazar $H_0$ si $|Z| > 1.812$ o si p-valor $< 0.070$
    \item Decisión: Rechazar $H_0$
    \item Conclusión: Con un nivel de significancia de 0.070, la evidencia sugiere que la proporción de personas que asistieron a reuniones familiares no es igual al 50\%.
\end{itemize}

\section*{Ejercicio 3: Prueba de hipótesis para la media de una muestra con varianza desconocida}
\begin{itemize}
    \item \textit{Planteamiento del problema:} La GEIH 2024 recopila información detallada sobre el mercado laboral colombiano, incluyendo las prácticas y pasantías como forma de vinculación laboral. La variable P3094S3 registra el valor mensual de estas actividades, las cuales son fundamentales para la transición de los jóvenes al mercado laboral formal. Considerando que las prácticas representan una oportunidad de formación y experiencia laboral, se cree que el valor promedio mensual por prácticas o pasantías es igual a 200,000 pesos.
    \item Hipótesis: \quad $H_{0}: \mu = 200,000$ \hspace{2cm} $H_{1}: \mu \neq 200,000$
    \item Nivel de significancia: $\alpha = 0.025$
    \item Estadístico de Prueba: t
\end{itemize}

\begin{tabular}{|m{7cm}|m{7cm}|}
\hline
\textbf{Estadístico Empleado} & t para media con varianza desconocida \\ \hline
\textbf{Fórmula del estadístico} & $t = \frac{\bar{X} - \mu_0}{s/\sqrt{n}}$ \\ \hline
\textbf{Nombre Parámetro} & \textbf{Valor} \\ \hline
Media poblacional bajo $H_0$ ($\mu_0$) & 200,000 \\ \hline
Tamaño de muestra (n) & 336 \\ \hline
Grados de libertad (df) & 335 \\ \hline
Media muestral ($\bar{X}$) & 998,708.33 \\ \hline
Desviación estándar muestral (s) & 621,977.40 \\ \hline
Valor Estadístico & 23.539 \\ \hline
Valor P & $< 0.001$ \\ \hline
Valor Crítico & $\pm 2.252$ \\ \hline
\end{tabular}

\begin{itemize}
    \item Regla de decisión: Rechazar $H_0$ si $|t| > 2.252$ o si p-valor $< 0.025$
    \item Decisión: Rechazar $H_0$
    \item Conclusión: Con un nivel de significancia de 0.025, la evidencia sugiere que el valor promedio mensual por prácticas no es igual a 200,000 pesos.
\end{itemize}

\section*{Ejercicio 4: Prueba de hipótesis para la diferencia de medias de dos muestras con varianzas conocidas}
\begin{itemize}
    \item \textit{Planteamiento del problema:} La GEIH 2024 documenta las actividades agropecuarias de los hogares colombianos, incluyendo tanto el cultivo (P3095S1) como la cría de animales (P3095S3). Estas actividades representan estrategias económicas diferentes: el cultivo requiere inversión en semillas y fertilizantes, mientras que la cría de animales implica costos de alimentación y cuidado. Ambas actividades son fundamentales para la seguridad alimentaria y los ingresos rurales. Se cree que no existe diferencia entre el ahorro promedio por cultivar y por criar animales.
    \item Hipótesis: \quad $H_{0}: \mu_1 - \mu_2 = 0$ \hspace{2cm} $H_{1}: \mu_1 - \mu_2 \neq 0$
    \item Nivel de significancia: $\alpha = 0.045$
    \item Estadístico de Prueba: Z
\end{itemize}

\begin{tabular}{|m{7cm}|m{7cm}|}
\hline
\textbf{Estadístico Empleado} & Z para diferencia de medias con varianzas conocidas \\ \hline
\textbf{Fórmula del estadístico} & $Z = \frac{(\bar{X}_1 - \bar{X}_2) - 0}{\sqrt{\sigma_1^2/n_1 + \sigma_2^2/n_2}}$ \\ \hline
\textbf{Nombre Parámetro} & \textbf{Valor} \\ \hline
Media muestra 1 (cultivar) & 126,937.56 \\ \hline
Media muestra 2 (criar) & 159,471.37 \\ \hline
Desviación estándar 1 ($\sigma_1$) & 300,000 \\ \hline
Desviación estándar 2 ($\sigma_2$) & 500,000 \\ \hline
Tamaño muestra 1 ($n_1$) & 2,892 \\ \hline
Tamaño muestra 2 ($n_2$) & 4,080 \\ \hline
Diferencia de medias & -32,533.81 \\ \hline
Valor Estadístico & -3.385 \\ \hline
Valor P & 0.0007 \\ \hline
Valor Crítico & $\pm 2.005$ \\ \hline
\end{tabular}

\begin{itemize}
    \item Regla de decisión: Rechazar $H_0$ si $|Z| > 2.005$ o si p-valor $< 0.045$
    \item Decisión: Rechazar $H_0$
    \item Conclusión: Con un nivel de significancia de 0.045, la evidencia sugiere que hay diferencia entre el ahorro promedio por cultivar y por criar animales.
\end{itemize}

\section*{Ejercicio 5: Prueba de hipótesis para la diferencia de medias de dos muestras con varianzas iguales y desconocidas}
\begin{itemize}
    \item \textit{Planteamiento del problema:} La GEIH 2024 permite comparar diferentes fuentes de ingresos: las prácticas laborales (P3094S3) representan ingresos del sector formal/educativo, mientras que el cultivo (P3095S1) representa ingresos del sector agropecuario. Esta comparación es relevante para entender las diferencias entre ingresos urbanos/educativos versus rurales/agropecuarios en Colombia. Se cree que no existe diferencia entre el valor promedio de prácticas y el ahorro promedio por cultivar, asumiendo varianzas iguales.
    \item Hipótesis: \quad $H_{0}: \mu_1 - \mu_2 = 0$ \hspace{2cm} $H_{1}: \mu_1 - \mu_2 \neq 0$
    \item Nivel de significancia: $\alpha = 0.001$
    \item Estadístico de Prueba: t (pooled)
\end{itemize}

\begin{tabular}{|m{7cm}|m{7cm}|}
\hline
\textbf{Estadístico Empleado} & t para diferencia de medias con varianzas iguales \\ \hline
\textbf{Fórmula del estadístico} & $t = \frac{(\bar{X}_1 - \bar{X}_2) - 0}{s_p\sqrt{1/n_1 + 1/n_2}}$ \\ \hline
\textbf{Nombre Parámetro} & \textbf{Valor} \\ \hline
Media muestra 1 (prácticas) & 998,708.33 \\ \hline
Media muestra 2 (cultivar) & 126,937.56 \\ \hline
Tamaño muestra 1 ($n_1$) & 336 \\ \hline
Tamaño muestra 2 ($n_2$) & 2,892 \\ \hline
Grados de libertad (df) & 3,226 \\ \hline
Varianza combinada ($s_p^2$) & 190,877,666,859 \\ \hline
Diferencia de medias & 871,770.77 \\ \hline
Valor Estadístico & 34.620 \\ \hline
Valor P & $< 0.001$ \\ \hline
Valor Crítico & $\pm 3.294$ \\ \hline
\end{tabular}

\begin{itemize}
    \item Regla de decisión: Rechazar $H_0$ si $|t| > 3.294$ o si p-valor $< 0.001$
    \item Decisión: Rechazar $H_0$
    \item Conclusión: Con un nivel de significancia de 0.001, la evidencia sugiere que hay diferencia entre el valor promedio de prácticas y el ahorro promedio por cultivar.
\end{itemize}

\section*{Ejercicio 6: Prueba de hipótesis para la diferencia de medias de dos muestras con varianzas desiguales y desconocidas}
\begin{itemize}
    \item \textit{Planteamiento del problema:} La GEIH 2024 permite analizar la variabilidad de ingresos entre diferentes actividades económicas. Las prácticas laborales (P3094S3) y la cría de animales (P3095S3) representan sectores económicos distintos con diferentes niveles de riesgo e incertidumbre. Las prácticas están más reguladas y estandarizadas, mientras que la cría de animales depende de factores climáticos y de mercado más variables. Se cree que no existe diferencia entre el valor promedio de prácticas y el ahorro promedio por criar animales, sin asumir varianzas iguales.
    \item Hipótesis: \quad $H_{0}: \mu_1 - \mu_2 = 0$ \hspace{2cm} $H_{1}: \mu_1 - \mu_2 \neq 0$
    \item Nivel de significancia: $\alpha = 0.030$
    \item Estadístico de Prueba: t de Welch
\end{itemize}

\begin{tabular}{|m{7cm}|m{7cm}|}
\hline
\textbf{Estadístico Empleado} & t de Welch para diferencia de medias \\ \hline
\textbf{Fórmula del estadístico} & $t = \frac{(\bar{X}_1 - \bar{X}_2) - 0}{\sqrt{s_1^2/n_1 + s_2^2/n_2}}$ \\ \hline
\textbf{Nombre Parámetro} & \textbf{Valor} \\ \hline
Media muestra 1 (prácticas) & 998,708.33 \\ \hline
Media muestra 2 (criar) & 159,471.37 \\ \hline
Tamaño muestra 1 ($n_1$) & 336 \\ \hline
Tamaño muestra 2 ($n_2$) & 4,080 \\ \hline
Grados de libertad (df) & 373.42 \\ \hline
Diferencia de medias & 839,236.96 \\ \hline
Error estándar & 34,867.44 \\ \hline
Valor Estadístico & 24.069 \\ \hline
Valor P & $< 0.001$ \\ \hline
Valor Crítico & $\pm 2.178$ \\ \hline
\end{tabular}

\begin{itemize}
    \item Regla de decisión: Rechazar $H_0$ si $|t| > 2.178$ o si p-valor $< 0.030$
    \item Decisión: Rechazar $H_0$
    \item Conclusión: Con un nivel de significancia de 0.030, la evidencia sugiere que hay diferencia entre el valor promedio de prácticas y el ahorro promedio por criar animales.
\end{itemize}

\section*{Ejercicio 7: Prueba de hipótesis para la diferencia de proporciones de dos muestras}
\begin{itemize}
    \item \textit{Planteamiento del problema:} La GEIH 2024 permite analizar la relación entre el nivel socioeconómico y la participación social familiar. Se han creado grupos según el valor de prácticas (P3094S3): bajo y alto valor, para examinar si el nivel de ingresos por prácticas se asocia con la participación en reuniones familiares (P3087S1). Esta relación es importante para entender cómo los ingresos laborales pueden influir en el capital social familiar. Se cree que no existe diferencia en la proporción de asistencia a reuniones familiares entre grupos de bajo y alto valor de prácticas.
    \item Hipótesis: \quad $H_{0}: p_1 - p_2 = 0$ \hspace{2cm} $H_{1}: p_1 - p_2 \neq 0$
    \item Nivel de significancia: $\alpha = 0.070$
    \item Estadístico de Prueba: Z
\end{itemize}

\begin{tabular}{|m{7cm}|m{7cm}|}
\hline
\textbf{Estadístico Empleado} & Z para diferencia de proporciones \\ \hline
\textbf{Fórmula del estadístico} & $Z = \frac{(\hat{p}_1 - \hat{p}_2) - 0}{\sqrt{\hat{p}(1-\hat{p})(1/n_1 + 1/n_2)}}$ \\ \hline
\textbf{Nombre Parámetro} & \textbf{Valor} \\ \hline
Proporción grupo bajo ($\hat{p}_1$) & 0.0138 \\ \hline
Proporción grupo alto ($\hat{p}_2$) & 0.0085 \\ \hline
Tamaño grupo bajo ($n_1$) & 145 \\ \hline
Tamaño grupo alto ($n_2$) & 118 \\ \hline
Proporción combinada ($\hat{p}$) & 0.0119 \\ \hline
Diferencia de proporciones & 0.0053 \\ \hline
Valor Estadístico & 0.427 \\ \hline
Valor P & 0.6697 \\ \hline
Valor Crítico & $\pm 1.812$ \\ \hline
\end{tabular}

\begin{itemize}
    \item Regla de decisión: Rechazar $H_0$ si $|Z| > 1.812$ o si p-valor $< 0.070$
    \item Decisión: No rechazar $H_0$
    \item Conclusión: Con un nivel de significancia de 0.070, la evidencia sugiere que no hay diferencia en la proporción de asistencia a reuniones familiares entre grupos de bajo y alto valor de prácticas.
\end{itemize}

\section*{Ejercicio 8: Prueba de hipótesis para muestras dependientes}
\begin{itemize}
    \item \textit{Planteamiento del problema:} La GEIH 2024 permite analizar hogares que realizan múltiples actividades agropecuarias simultáneamente. Muchos hogares rurales colombianos combinan el cultivo (P3095S1) con la cría de animales (P3095S3) como estrategia de diversificación de ingresos y reducción de riesgos. Esta diversificación es común en la agricultura familiar colombiana. Se cree que no existe diferencia sistemática entre el ahorro por cultivar y por criar animales en las mismas personas.
    \item Hipótesis: \quad $H_{0}: \mu_d = 0$ \hspace{2cm} $H_{1}: \mu_d \neq 0$
    \item Nivel de significancia: $\alpha = 0.025$
    \item Estadístico de Prueba: t pareado
\end{itemize}

\begin{tabular}{|m{7cm}|m{7cm}|}
\hline
\textbf{Estadístico Empleado} & t para muestras pareadas \\ \hline
\textbf{Fórmula del estadístico} & $t = \frac{\bar{d} - 0}{s_d/\sqrt{n}}$ \\ \hline
\textbf{Nombre Parámetro} & \textbf{Valor} \\ \hline
Número de pares (n) & 1,296 \\ \hline
Grados de libertad (df) & 1,295 \\ \hline
Media de diferencias ($\bar{d}$) & 9,776.68 \\ \hline
Desviación estándar de diferencias ($s_d$) & 575,782.40 \\ \hline
Error estándar de diferencias & 15,993.96 \\ \hline
Valor Estadístico & 0.611 \\ \hline
Valor P & 0.5411 \\ \hline
Valor Crítico & $\pm 2.244$ \\ \hline
\end{tabular}

\begin{itemize}
    \item Regla de decisión: Rechazar $H_0$ si $|t| > 2.244$ o si p-valor $< 0.025$
    \item Decisión: No rechazar $H_0$
    \item Conclusión: Con un nivel de significancia de 0.025, la evidencia sugiere que no hay diferencia entre el ahorro por cultivar y por criar animales en las mismas personas.
\end{itemize}

\section*{Ejercicio 9: Prueba de bondad de ajuste}
\begin{itemize}
    \item \textit{Planteamiento del problema:} La GEIH 2024 recopila información sobre la participación social familiar a través de la variable P3087S1 (asistencia a reuniones familiares). En Colombia, las reuniones familiares son eventos culturalmente importantes que reflejan la cohesión social. Una distribución uniforme (50\%-50\%) representaría una participación equilibrada, pero factores como la urbanización, migración y cambios en la estructura familiar pueden influir en esta participación. Se cree que la distribución observada de asistencia a reuniones familiares se ajusta a una distribución uniforme (50\%-50\%).
    \item Hipótesis: \quad $H_{0}:$ La distribución observada se ajusta a la esperada \hspace{1cm} $H_{1}:$ La distribución observada no se ajusta a la esperada
    \item Nivel de significancia: $\alpha = 0.045$
    \item Estadístico de Prueba: $\chi^2$
\end{itemize}

\begin{tabular}{|m{7cm}|m{7cm}|}
\hline
\textbf{Estadístico Empleado} & Chi-cuadrado de bondad de ajuste \\ \hline
\textbf{Fórmula del estadístico} & $\chi^2 = \sum \frac{(O_i - E_i)^2}{E_i}$ \\ \hline
\textbf{Nombre Parámetro} & \textbf{Valor} \\ \hline
Número de categorías (k) & 2 \\ \hline
Grados de libertad (df) & 1 \\ \hline
Frecuencia observada "No" ($O_1$) & 163,423 \\ \hline
Frecuencia observada "Sí" ($O_2$) & 2,918 \\ \hline
Frecuencia esperada "No" ($E_1$) & 83,170.5 \\ \hline
Frecuencia esperada "Sí" ($E_2$) & 83,170.5 \\ \hline
Valor Estadístico & 154,873.75 \\ \hline
Valor P & $< 0.001$ \\ \hline
Valor Crítico & 4.019 \\ \hline
\end{tabular}

\begin{itemize}
    \item Regla de decisión: Rechazar $H_0$ si $\chi^2 > 4.019$ o si p-valor $< 0.045$
    \item Decisión: Rechazar $H_0$
    \item Conclusión: Con un nivel de significancia de 0.045, la evidencia sugiere que la distribución observada de asistencia a reuniones familiares no se ajusta a una distribución uniforme.
\end{itemize}

\section*{Ejercicio 10: Prueba de independencia}
\begin{itemize}
    \item \textit{Planteamiento del problema:} La GEIH 2024 permite examinar la relación entre variables socioeconómicas y participación social. Se ha categorizado el valor de prácticas (P3094S3) en terciles (bajo, medio, alto) para analizar su asociación con la asistencia a reuniones familiares (P3087S1). Esta relación es relevante para entender si los ingresos laborales se asocian con la participación en actividades familiares, lo cual puede reflejar diferencias en capital social según el nivel socioeconómico. Se cree que no existe asociación entre la asistencia a reuniones familiares y el nivel de valor de prácticas (agrupado en terciles).
    \item Hipótesis: \quad $H_{0}:$ Las variables son independientes \hspace{2cm} $H_{1}:$ Las variables están asociadas
    \item Nivel de significancia: $\alpha = 0.001$
    \item Estadístico de Prueba: $\chi^2$
\end{itemize}

\begin{tabular}{|m{7cm}|m{7cm}|}
\hline
\textbf{Estadístico Empleado} & Chi-cuadrado de independencia \\ \hline
\textbf{Fórmula del estadístico} & $\chi^2 = \sum \frac{(O_{ij} - E_{ij})^2}{E_{ij}}$ \\ \hline
\textbf{Nombre Parámetro} & \textbf{Valor} \\ \hline
Número de filas (r) & 3 \\ \hline
Número de columnas (c) & 2 \\ \hline
Grados de libertad (df) & 2 \\ \hline
Frecuencias observadas & Bajo: 125,0; 2,1 \\ \hline
& Medio: 101,0; 1,1 \\ \hline
& Alto: 106,0; 1,1 \\ \hline
Frecuencias esperadas & Bajo: 125.5, 1.5 \\ \hline
& Medio: 100.8, 1.2 \\ \hline
& Alto: 105.7, 1.3 \\ \hline
Valor Estadístico & 0.257 \\ \hline
Valor P & 0.8793 \\ \hline
Valor Crítico & 13.816 \\ \hline
\end{tabular}

\begin{itemize}
    \item Regla de decisión: Rechazar $H_0$ si $\chi^2 > 13.816$ o si p-valor $< 0.001$
    \item Decisión: No rechazar $H_0$
    \item Conclusión: Con un nivel de significancia de 0.001, la evidencia sugiere que no hay asociación entre la asistencia a reuniones familiares y el nivel de valor de prácticas.
\end{itemize}

\section*{Ejercicio 11: Prueba de signos}
\begin{itemize}
    \item \textit{Planteamiento del problema:} La GEIH 2024 permite analizar la consistencia en el rendimiento de diferentes actividades agropecuarias a nivel de hogar. Se han creado rankings para el ahorro por cultivar (P3095S1) y por criar animales (P3095S3) para examinar si los hogares que tienen mejor rendimiento en una actividad también lo tienen en la otra. Esta relación es importante para entender si existe complementariedad o competencia entre estas actividades en los hogares rurales colombianos. Se cree que no existe tendencia sistemática en las diferencias entre rankings de ahorro por cultivar y por criar animales.
    \item Hipótesis: \quad $H_{0}:$ No hay diferencia sistemática de signos \hspace{1cm} $H_{1}:$ Existe diferencia sistemática de signos
    \item Nivel de significancia: $\alpha = 0.030$
    \item Estadístico de Prueba: Prueba de signos
\end{itemize}

\begin{tabular}{|m{7cm}|m{7cm}|}
\hline
\textbf{Estadístico Empleado} & Prueba de signos para muestras pareadas \\ \hline
\textbf{Fórmula del estadístico} & Estadístico = min(pos, neg) \\ \hline
\textbf{Nombre Parámetro} & \textbf{Valor} \\ \hline
Número total de pares (n) & 1,295 \\ \hline
Diferencias positivas & 679 \\ \hline
Diferencias negativas & 616 \\ \hline
Empates & 1 \\ \hline
Estadístico (min) & 616 \\ \hline
Valor P & 0.0849 \\ \hline
\end{tabular}

\begin{itemize}
    \item Regla de decisión: Rechazar $H_0$ si p-valor $< 0.030$
    \item Decisión: No rechazar $H_0$
    \item Conclusión: Con un nivel de significancia de 0.030, la evidencia sugiere que no hay tendencia en las diferencias entre rankings de ahorro por cultivar y por criar animales.
\end{itemize}

\end{document}